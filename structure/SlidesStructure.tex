%----------------------------------------------------------------------------------------
%	Codification text & symbols
%----------------------------------------------------------------------------------------
\usepackage[utf8]{inputenc} % Required for including letters with accents
\usepackage[T1]{fontenc} % Use 8-bit encoding that has 256 glyphs
\usepackage[spanish]{babel}			% Español %\usepackage[spanish]{layout}
\usepackage{Structure/ru}
%----------------------------------------------------------------------------------------
%	Varios Required Packages
%----------------------------------------------------------------------------------------
\usepackage{lipsum} % Inserts dummy text 
\usepackage{etoolbox} %Conditional macros
\usepackage{comment}
%\usepackage{blindtext}  % Inserts random paragraph \blindtext
%\pagestyle{empty} %Changes the style from the current page	plain empty eadings myheadings
%----------------------------------------------------------------------------------------
%	Colours Packages
%----------------------------------------------------------------------------------------
%\usepackage[table,x11names,dvipsnames,table]{xcolor} %\usepackage{xcolor}%Specifying colors by name
%\usepackage[colorlinks=true, urlcolor=blue]{hyperref} % Forinserting hyperlinks (colours)
\usepackage{tcolorbox} % Coloured boxes, for LATEX examples and theorems, etc
%----------------------------------------------------------------------------------------
%	Text and fonts
%----------------------------------------------------------------------------------------
\usepackage{textcomp} %for many-symbol
%\usepackage{titlesec} % Allows customization of titles
\usepackage{verbatim} % VERBATIM PACKAGE
\usepackage{microtype} % Slightly tweak font spacing for aesthetics
\usepackage{avant} % Use the Avantgarde font for headings
\RequirePackage{fix-cm} % permit Computer Modern fonts at arbitrary sizes.
\usepackage{calc} % For simpler calculation - used for spacing the index letter headings correctly
\usepackage[footnotesize,hang]{caption} % réduire la taille des légendes des images
\usepackage{times} % Use the Times font for headings
\setbeamertemplate{caption}[numbered]
%\usepackage{kpfonts}
%\usepackage{lmodern}
%----------------------------------------------------------------------------------------
%	Tables
%----------------------------------------------------------------------------------------
\usepackage{threeparttable} % tables with footnotes, capions all the same width
\usepackage{dcolumn}        % decimal-aligned tabular math columns
\usepackage{multirow}       % Allow table cells to span multiple rows
\usepackage{booktabs}       % Publication-quality tables -nicer horizontal rules in tables
\usepackage{ltxtable}       % long tabularx
\usepackage{colortbl}		% colured ‘panels’ behind specified columns in a table.
\usepackage{makecell}		% Celdas con diagonales
\usepackage{multicol}
\usepackage{tabularx}
%----------------------------------------------------------------------------------------
%	Image packages
%----------------------------------------------------------------------------------------
\usepackage{tikz} % Required for drawing custom shapes
\usetikzlibrary{matrix}
\usetikzlibrary{circuits}
\usetikzlibrary{arrows,shapes}
\usetikzlibrary{backgrounds}
\usepackage{graphicx} % Required for including pictures
\usepackage{subfig} % Required for creating figures with multiple parts (subfigures)
\usepackage{float}
\usepackage{placeins}
\usepackage{color}
\usepackage[siunitx]{circuitikz}
\usepackage{schemabloc}
\usepackage{eso-pic} % Required for specifying an image background in the title page
\usepackage{epstopdf}
%----------------------------------------------------------------------------------------
%	MATH PACKAGE
%----------------------------------------------------------------------------------------
\usepackage{amsmath,amssymb,amsthm} % For including math equations, theorems, symbols, etc
\usepackage{eqnarray}
\usepackage{cases} % function defined piecewise
\usepackage{mathtools}
\usepackage{mathptmx} % Use the Adobe Times Roman as the default text font together with math symbols from the Sym­bol, Chancery and Com­puter Modern fonts
\usepackage{amsfonts}
\usepackage{mathptmx}
\usepackage{bm}  % bold math symbols $\alpha \not= \bm{\alpha}$
\usepackage{array}
%%\usepackage{achemso} %Quimica
%%\usepackage{newtxmath} %Solo PDF
%%\usepackage{newtxtext} %Solo PDF
%----------------------------------------------------------------------------------------
%	Bibliography, References and Index
%----------------------------------------------------------------------------------------
\usepackage{makeidx} % Required to make an index
\makeindex % Tells LaTeX to create the files required for indexing
%\usepackage[dcucite]{harvard} %bibliography style, dcu gives commas before year, semicolon between references, and "and" between authors
%\usepackage{html}%to get harvard to work, insert immediately before \usepackage{harvard}
\usepackage{url}%to get harvard to work, insert immediately before \title{...}
\usepackage{cleveref} %Dice donde esta lo referenciado
\usepackage{varioref} % More descriptive referencing
\parindent=0pt
\usepackage{cite}
\hyphenation{Fortran hy-phen-ation} % Specify custom hyphenation points in words with dashes where you would

\usepackage{hyperref} %\usepackage{graphicx}% http://ctan.org/pkg/graphicx \usepackage{hyperref}
\tcbuselibrary{skins}
\usepackage{ragged2e}
\usepackage{booktabs}
\usepackage{listings}
\usepackage{helvet}
\usepackage{pdfpages}
%\setbeameroption{show notes}
%\setbeameroption{show notes on second screen=right}
\usefonttheme{professionalfonts}
\usepackage{cmap}
%\setbeamercovered{dynamic}
%\useinnertheme{rectangles}
%\definecolor{ocre}{RGB}{52,177,201}
%\DeclareCaptionFont{ocre}{\color{ocre}}
%\setbeamertemplate{footline}[frame number]
